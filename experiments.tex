\section{Experiments}\label{sec:experiments}


In this section, we first evaluate the proposed CME module as part of the architectures described above for two tasks: text-video retrieval and very-large scale action recognition. We then conduct an ablation study to understand the contribution of different components of our model. how the performance of the proposed approach is affected by different CMF designs. 


\subsection{Video Retrieval}

\subsubsection{Dataset and Implementation Details}

We perform experiments on four video datasets for retrieval:  MSR-VTT, LSMDC, DiDeMo and ActivityNet-captions, each of which is described briefly below.


\paragraph{ActivityNet-captions:} \cite{krishna2017dense} consists of a collection of 20k videos sourced from YouTube.  Each video in the dataset is accompanied by a descriptive sentences, numbering 100k in total.  For fair comparison, we adopt the retrieval evaluations and protocols used in prior work \cite{zhang2018cross,liu2019use}, training up to 15 epochs on the training partition (10,009 videos in total) and evaluating on the larger \texttt{val1} (4,917 videos in total).  We compare to the strongest results reported in the literature in Tab.~\ref{table:activity-net}. We observe that the 


\begin{table*}[h]
% \captionsetup{font=footnotesize}
\centering 
\footnotesize 
\setlength{\tabcolsep}{5pt}
\begin{tabular}{l | c c c c c | c c c c c} 
\hline \hline 
\multicolumn{1}{c}{} & 
\multicolumn{5}{c}{Text $\implies$ Video} & \multicolumn{5}{c}{Video $\implies$ Text} \\
Method & R$@$1 & R$@$5 & R$@$50 & MdR & MnR & R$@$1 & R$@$5 & R$@$50 & MdR & MnR \\ 
\hline 
LSTM-YT & 0.0 & 4.0 & 24.0 & 102 & - & 0.0 & 7.0 & 38.0  & 98 & - \\
NOCTXT & 5.0 & 14.0 & 32.0 & 78 & - & 7.0 & 18.0 & 45.0 & 56 & - \\

DENSE & 14.0 & 32.0 & 65.0 & 34 & - & 18.0 & 36.0 & 74.0 & 32 \\
HSE(4SEGS) & 20.5 & 49.3 & - & - & - & 18.7 & 48.1 & - & - \\
FSE & $\prepad 18.2_{\pm0.2}$ & $\prepad 44.8_{\pm0.4}$ & $\prepad {89.1}_{\pm 0.3}$ & $7$ & - & $\prepad {16.7}_{\pm0.8}$ & $\prepad {43.1}_{\pm1.1}$ & $\prepad 88.4_{\pm0.3}$ & $7$ &  - \\
CE & $\prepad 27.3_{\pm0.7}$ & $\prepad 61.1_{\pm1.0}$ & $\prepad 94.4_{\pm0.1}$ & $\prepadmini 4_{\pm0}$ & $\prepad 15.4_{\pm0.9}$ 
& $\prepad 27.9_{\pm0.6}$ & $\prepad 61.6_{\pm0.4}$ & $\prepad 95.0_{\pm0.2}$ & $\prepadmini 3.3_{\pm0.6}$ & $\prepad 14.5_{\pm0.8}$ \\
\hline
% CMF-pairwise & $\prepad 31.6_{\pm0.3}$ & $\prepad 67.2_{\pm0.2}$ & $\prepad 97.0_{\pm0.2}$ & $\prepad 3.0_{\pm0.0}$ & $\prepad 9.9_{\pm0.1}$ & $\prepad 32.0_{\pm0.4}$ & $\prepad 68.9_{\pm1.1}$ & $\prepad 97.1_{\pm0.3}$ & $\prepad 3.0_{\pm0.0}$ & $\prepad 9.5_{\pm0.4}$ \\
% CMF-pairwise-star & $\prepad 31.7_{\pm0.5}$ & $\prepad 66.9_{\pm0.6}$ & $\prepad 96.7_{\pm0.3}$ & $\prepad 3.0_{\pm0.0}$ & $\prepad 10.0_{\pm0.3}$ & $\prepad 32.7_{\pm0.4}$ & $\prepad 69.2_{\pm0.8}$ & $\prepad 96.9_{\pm0.1}$ & $\prepad 3.0_{\pm0.0}$ & $\prepad 9.4_{\pm0.3}$ \\
CMF-pairwise-star-tensor & $\prepad 32.3_{\pm0.3}$ & $\prepad 67.6_{\pm0.3}$ & $\prepad 97.2_{\pm0.1}$ & $\prepad 3.0_{\pm0.0}$ & $\prepad 9.50_{\pm0.2}$ & $\prepad 33.5_{\pm0.7}$ & $\prepad 69.2_{\pm1.2}$ & $\prepad 97.4_{\pm0.1}$ & $\prepad 3.0_{\pm0.0}$ & $\prepad 8.9_{\pm0.6}$ \\
% CMF-triplet & $\prepad 30.5_{\pm0.2}$ & $\prepad 66.5_{\pm0.7}$ & $\prepad 96.8_{\pm0.1}$ & $\prepad 3.0_{\pm0.0}$ & $\prepad 9.8_{\pm0.2}$ & $\prepad 32.0_{\pm1.4}$ & $\prepad 67.5_{\pm0.8}$ & $\prepad 96.8_{\pm0.4}$ & $\prepad 3.0_{\pm0.0}$ & $\prepad 9.4_{\pm0.6}$ \\
% % CMF-pairwise-star-random & $\prepad 27.4_{\pm0.7}$ & $\prepad 62.6_{\pm1.4}$ & $\prepad 96.1_{\pm0.1}$ & $\prepad 3.0_{\pm0.0}$ & $\prepad 11.3_{\pm0.3}$ & $\prepad 29.6_{\pm0.6}$ & $\prepad 65.5_{\pm1.0}$ & $\prepad 96.8_{\pm0.1}$ & $\prepad 3.0_{\pm0.0}$ & $\prepad 10.2_{\pm0.2}$ \\
% CMF-pairwise-star-specific & $\prepad 31.1_{\pm0.7}$ & $\prepad 65.9_{\pm1.4}$ & $\prepad 96.8_{\pm0.1}$ & $\prepad 3.0_{\pm0.0}$ & $\prepad 10.1_{\pm0.3}$ & $\prepad 31.9_{\pm0.6}$ & $\prepad 67.6_{\pm1.0}$ & $\prepad 96.9_{\pm0.1}$ & $\prepad 3.0_{\pm0.0}$ & $\prepad 9.5_{\pm0.2}$ \\

\hline \hline
\end{tabular}
% \vspace{0.3cm}
\caption{\color{red}Comparison of paragraph-video retrieval methods trained with video-level information on the ActivityNet-captions dataset (val1 test-split). For reference: LSTM-YT~\cite{venugopalan2015sequence} (\cite{zhang2018cross}), NOCTXT~\cite{venugopalan2014translating} (\cite{zhang2018cross}), HSE(4SEGS)~\cite{zhang2018cross}, DENSE~\cite{krishna2017dense}, FSE~\cite{zhang2018cross} \color{black}}
\label{table:activity-net} 
\end{table*}

\paragraph{DiDeMo:} \cite{anne2017localizing} consists of a set of personal videos, captured in a diverse array of visual settings (10,464 videos in total).  We follow prior work on \textit{video-level} text-video retrieval (we do not make use of timestamp information during training or testing).  We compare to the existing state-of-the-art in Tab.~\ref{table:DiDeMo}.

\begin{table*}[h!]
\centering 
\footnotesize 
\setlength{\tabcolsep}{2pt}
\begin{tabular}{l | c c c c c | c c c c c} 
\hline \hline
\multicolumn{1}{c}{} & 
\multicolumn{5}{c}{Text $\implies$ Video} & \multicolumn{5}{c}{Video $\implies$ Text} \\
Method & R$@$1 & R$@$5 & R$@$50 & MdR & MnR & R$@$1 & R$@$5 & R$@$50 & MdR & MnR \\ 
\hline 
S2VT  & 11.9 & 33.6 & 76.5 & 13 & - & 13.2 & 33.6 & 76.5 & 15 & - \\
FSE & $\prepad 13.9_{\pm0.7}$ & $\prepad 36_{\pm0.8}$ & $\prepad {78.9}_{\pm 1.6}$ & $11$ & - & $\prepad {13.1}_{\pm0.5}$ & $\prepad {33.9}_{\pm0.4}$ & $\prepad 78.0_{\pm0.8}$ & $12$ &  - \\
CE & $\prepad 22.6_{\pm0.5}$ & $\prepad 51.1_{\pm1.0}$ & $\prepad 87.3_{\pm0.3}$ & $\hspace{1em} 5_{\pm0}$ & $\prepad 27.4_{\pm1.1}$ & $\prepad 22.5_{\pm1.3}$ & $\prepad 52.3_{\pm0.8}$ & $\prepad 89.2_{\pm0.8}$ & $\hspace{1em} 5_{\pm0}$ & $\prepad 23.2_{\pm1.1}$ \\
\hline
% CMF-pairwise-star & $\prepad 25.0_{\pm0.7}$ & $\prepad 54.8_{\pm0.4}$ & $\prepad 91.0_{\pm1.3}$ & $\prepad 4.3_{\pm0.6}$ & $\prepad 17.3_{\pm0.5}$ & $\prepad 24.4_{\pm1.0}$ & $\prepad 55.2_{\pm1.4}$ & $\prepad 93.7_{\pm0.4}$ & $\prepad 4.0_{\pm0.0}$ & $\prepad 14.0_{\pm0.4}$ \\
% CMF-pairwise-noMOE & $\prepad 25.4_{\pm1.3}$ & $\prepad 54.5_{\pm0.4}$ & $\prepad 91.8_{\pm0.3}$ & $\prepad 4.3_{\pm0.6}$ & $\prepad 17.1_{\pm0.3}$ & $\prepad 25.1_{\pm0.8}$ & $\prepad 56.1_{\pm1.0}$ & $\prepad 93.4_{\pm0.8}$ & $\prepad 4.0_{\pm0.0}$ & $\prepad 14.0_{\pm0.5}$ \\
CMF-pairwise-star-tensor & $\prepad 25.4_{\pm0.4}$ & $\prepad 54.2_{\pm0.3}$ & $\prepad 91.6_{\pm0.7}$ & $\prepad 4.7_{\pm0.6}$ & $\prepad 17.3_{\pm0.9}$ & $\prepad 25.4_{\pm1.2}$ & $\prepad 56.0_{\pm0.4}$ & $\prepad 93.8_{\pm1.1}$ & $\prepad 4.0_{\pm0.0}$ & $\prepad 14.0_{\pm0.6}$ \\
% CMF-triplet & $\prepad 24.6_{\pm0.7}$ & $\prepad 55.0_{\pm0.9}$ & $\prepad 91.7_{\pm0.6}$ & $\prepad 4.3_{\pm0.6}$ & $\prepad 16.7_{\pm0.8}$ & $\prepad 26.1_{\pm0.6}$ & $\prepad 56.0_{\pm2.0}$ & $\prepad 93.6_{\pm0.8}$ & $\prepad 4.3_{\pm0.6}$ & $\prepad 13.7_{\pm0.8}$ \\
% CMF-pairwise-star-avgmax & $\prepad 24.2_{\pm0.5}$ & $\prepad 55.0_{\pm0.6}$ & $\prepad 91.8_{\pm0.4}$ & $\prepad 4.3_{\pm0.6}$ & $\prepad 16.8_{\pm0.4}$ & $\prepad 25.6_{\pm0.9}$ & $\prepad 55.5_{\pm1.4}$ & $\prepad 93.8_{\pm0.4}$ & $\prepad 4.3_{\pm0.6}$ & $\prepad 14.1_{\pm0.4}$ \\
% CMF-pairwise-star-random & $\prepad 15.9_{\pm8.5}$ & $\prepad 37.5_{\pm12.8}$ & $\prepad 80.5_{\pm9.3}$ & $\prepad 11.3_{\pm5.5}$ & $\prepad 36.5_{\pm15.2}$ & $\prepad 15.1_{\pm10.0}$ & $\prepad 36.6_{\pm15.2}$ & $\prepad 80.0_{\pm11.1}$ & $\prepad 12.3_{\pm7.2}$ & $\prepad 38.1_{\pm20.7}$ \\

\hline \hline
\end{tabular}
\caption{\color{red}Paragraph-video retrieval comparison of  methods trained with video-level information on the DiDeMo benchmark. For reference: SV2T~\cite{venugopalan2014translating} (\cite{zhang2018cross}), FSE~\cite{zhang2018cross}\color{black}}
\label{table:DiDeMo} 
\end{table*}

\paragraph{MSR-VTT:} \cite{xu2016msr} consists of videos sourced from YouTube paired with human-annotator captions, numbering 200k unique video-caption pairs in total. The dataset provides 6513 training, 497 validation and 2990 test videos.    We closely follow the evaluation protocol followed in \cite{liu2019use} to enable a fair comparison and report results in Tab.~\ref{table:MSRVTT}. 

\begin{table*}[t]
\centering 
\footnotesize 
% \setlength{\tabcolsep}{2pt}
\begin{tabular}{l | c | c c c c c |@{\hskip -0.2cm}c@{\hskip -0.35cm}c@{\hskip -0.35cm}c@{\hskip -0.1cm}c@{\hskip -0.2cm}c} 
\hline \hline
\multicolumn{2}{c}{} & 
\multicolumn{5}{c}{Text $\implies$ Video} & \multicolumn{5}{c}{Video $\implies$ Text} \\
Method  & R$@$1 & R$@$5 & R$@$10 & MdR & MnR & R$@$1 & R$@$5 & R$@$10 & MdR & MnR \\ 
\hline 
% \textbf{HOI}  & 1k-B & $\prepad \mathbf{00.0}_{\pm0.0}$  & $\prepad \mathbf{00.0}_{\pm0.0}$  &  $\prepad \mathbf{00.0}_{\pm0.0}$  &  $\prepadmini \mathbf{0}_{\pm0}$ & $\prepad \mathbf{0.0}_{\pm0.0}$ & $\prepad \mathbf{00.0}_{\pm0.0}$ & $\prepad \mathbf{00.0}_{\pm0.0}$  & $\prepad \mathbf{00.0}_{\pm0.0}$  &  $\prepadmini \mathbf{0}_{\pm0}$ & $\prepad \mathbf{0.0}_{\pm0.0}$ \\
% \textbf{AHOI}  & 1k-B & $\prepad \mathbf{00.0}_{\pm0.0}$  & $\prepad \mathbf{00.0}_{\pm0.0}$  &  $\prepad \mathbf{00.0}_{\pm0.0}$  &  $\prepadmini \mathbf{0}_{\pm0}$ & $\prepad \mathbf{0.0}_{\pm0.0}$ & $\prepad \mathbf{00.0}_{\pm0.0}$ & $\prepad \mathbf{00.0}_{\pm0.0}$  & $\prepad \mathbf{00.0}_{\pm0.0}$  &  $\prepadmini \mathbf{0}_{\pm0}$ & $\prepad \mathbf{0.0}_{\pm0.0}$ \\
\hline 
VSE  & 5.0 & 16.4 & 24.6 & 47 & 215.1 & 7.7 & 20.3 & 31.2 & 28 & 185.8\\
VSE++ & 5.7 & 17.1 & 24.8 & 65 & 300.8 & 10.2 & 25.4 & 35.1 & 25 & 228.1 \\
Mithun et al. &  7.0 & 20.9 & 29.7 & 38 & 213.8  & 12.5 & 32.1 & 42.4 & 16 & 134.0 \\
W2VV & 6.1 & 18.7 & 27.5 & 45 & - &  11.8 & 28.9 & 39.1 & 21 & -  \\  
Dual Encoding & 7.7 & 22.0 & 31.8 & 32 & - &  13.0 & 30.8 & 43.3 & 15 & - \\
\textbf{CE}  & $\prepad \mathbf{22.5}_{\pm0.1}$ & $\prepad \mathbf{52.1}_{\pm0.2}$ & $\prepad \mathbf{65.5}_{\pm0.1}$ & $\prepadmini \mathbf{5}_{\pm0}$ & $\quad \mathbf{22.5}_{\pm0.1}$ & $\quad \mathbf{34.4}_{\pm0.4}$ & $\prepad \mathbf{64.6}_{\pm0.3}$ & $\prepad \mathbf{77.0}_{\pm0.4}$ & $\prepadmini \mathbf{3}_{\pm0}$ & $\prepad \mathbf{13.2}_{\pm0.6}$ \\
CME-pairwise & $\prepad 23.2_{\pm0.2}$ & $\prepad 54.7_{\pm0.3}$ & $\prepad 68.8_{\pm0.3}$ & $\prepad 4.7_{\pm0.6}$ & $\prepad 17.1_{\pm0.1}$ & $\prepad 32.5_{\pm0.4}$ & $\prepad 67.5_{\pm0.4}$ & $\prepad 80.1_{\pm0.2}$ & $\prepad 3.0_{\pm0.0}$ & $\prepad 9.7_{\pm0.4}$ \\
CME-pairwise-star & $\prepad 22.9_{\pm0.3}$ & $\prepad 54.3_{\pm0.3}$ & $\prepad 68.6_{\pm0.3}$ & $\prepad 5.0_{\pm0.0}$ & $\prepad 17.2_{\pm0.1}$ & $\prepad 33.4_{\pm0.5}$ & $\prepad 67.9_{\pm0.2}$ & $\prepad 80.1_{\pm0.6}$ & $\prepad 3.0_{\pm0.0}$ & $\prepad 9.6_{\pm0.3}$ \\
CMF-pairwise-star-tensor & $\prepad 23.0_{\pm0.1}$ & $\prepad 54.5_{\pm0.3}$ & $\prepad 68.6_{\pm0.3}$ & $\prepad 4.7_{\pm0.6}$ & $\prepad 17.2_{\pm0.2}$ & $\prepad 33.2_{\pm0.6}$ & $\prepad 68.2_{\pm1.1}$ & $\prepad 80.6_{\pm0.7}$ & $\prepad 2.8_{\pm0.3}$ & $\prepad 9.5_{\pm0.2}$ \\
CME-pairwise-star-specific & $\prepad 22.9_{\pm0.1}$ & $\prepad 54.4_{\pm0.4}$ & $\prepad 68.8_{\pm0.1}$ & $\prepad 4.7_{\pm0.6}$ & $\prepad 17.1_{\pm0.1}$ & $\prepad 33.6_{\pm1.1}$ & $\prepad 67.4_{\pm0.3}$ & $\prepad 79.7_{\pm0.4}$ & $\prepad 3.0_{\pm0.0}$ & $\prepad 9.8_{\pm0.4}$ \\
% \textbf{HOI}  & Full & $\prepad \mathbf{00.0}_{\pm0.0}$  & $\prepad \mathbf{00.0}_{\pm0.0}$  &  $\prepad \mathbf{00.0}_{\pm0.0}$  &  $\prepadmini \mathbf{0}_{\pm0}$ & $\prepad \mathbf{0.0}_{\pm0.0}$ & $\prepad \mathbf{00.0}_{\pm0.0}$ & $\prepad \mathbf{00.0}_{\pm0.0}$  & $\prepad \mathbf{00.0}_{\pm0.0}$  &  $\prepadmini \mathbf{0}_{\pm0}$ & $\prepad \mathbf{0.0}_{\pm0.0}$ \\
% \textbf{AHOI}  & Full & $\prepad \mathbf{00.0}_{\pm0.0}$  & $\prepad \mathbf{00.0}_{\pm0.0}$  &  $\prepad \mathbf{00.0}_{\pm0.0}$  &  $\prepadmini \mathbf{0}_{\pm0}$ & $\prepad \mathbf{0.0}_{\pm0.0}$ & $\prepad \mathbf{00.0}_{\pm0.0}$ & $\prepad \mathbf{00.0}_{\pm0.0}$  & $\prepad \mathbf{00.0}_{\pm0.0}$  &  $\prepadmini \mathbf{0}_{\pm0}$ & $\prepad \mathbf{0.0}_{\pm0.0}$ \\
\hline \hline
\end{tabular}
\vspace{0.2cm}
\caption{\color{red}Retrieval using sentences and videos on the MSR-VTT dataset. R$@$k denotes recall$@$k (higher is better), MdR and MnR denote median rank and mean rank resp. (lower is better). 1k-A and 1k-B denote test sets of 1000 randomly sampled text-video pairs used by~\cite{yu2018joint} and~\cite{miech2018learning} resp.  We report standard deviations from three randomly seeded runs.  $\ddagger$~Updated code-base released by Miech et al.~\cite{miech2018learning} which achieves stronger results. For reference: JSFusion~\cite{yu2018joint}, MoEE~\cite{miech2018learning}, VSE~\cite{mithun2018learning}, VSE++ \cite{mithun2018learning}, W2VV~\cite{dong2018predicting}, Dual Encoding~\cite{dong2018dual} \color{black}}
\label{table:MSRVTT} 
\end{table*}

\normalsize

\paragraph{LSMDC:} \cite{rohrbach2015dataset} is a dataset collected using both movie scripts and transcribed DVS (descriptive video services that are provided for the visually impaired).  The dataset contains 118,081 short video clips in total. To compare with prior work, we report results in Tab.~\ref{table:LSMDC-MSVD}  on the 1000 video test set prescribed by the Large Scale Movie Description Challenge (LSMDC).

\begin{table*}[ht]
\centering 
\begin{minipage}{0.45\linewidth}
\footnotesize
\setlength{\tabcolsep}{5pt}
\hspace{-5em}
\begin{tabular}{l | c c c c }
\hline \hline
\multicolumn{1}{c}{} & 
\multicolumn{4}{c}{Text $\implies$ Video} \\
Method & R$@$1 & R$@$5 & R$@$10 & MdR  \\ 
\hline 
Yu et al.$\dagger$ &3.6 & 14.7 & 23.9 & 50 \\
CCA  &7.5 & 21.7 & 31.0 & 33  \\
JSFusion $\ddagger$  & 9.1 & 21.2 & 34.1 & 36  \\
MoEE &  11.6 & 28.0 & 37.6 & 22   \\
$\text{MoEE}_\text{COCO}$ &  12.7 & 28.9 & 39.6 & 21   \\
\textbf{CE}  & $\prepad 12.6_{\pm0.8}$ & $\prepad 31.0_{\pm0.6}$ & $\prepad 40.3_{\pm0.5}$ & $\hspace{0.8em} 19_{\pm0}$  \\
\hline
No-CMF & \prepad 15.0_{\pm0.2} & \prepad 32.5_{\pm0.6} & \prepad 42.9_{\pm0.8} & \prepad 16.0_{\pm0.0} \\
CMF-pairwise & \prepad 17.2_{\pm0.4} & \prepad 37.6_{\pm0.6} & \prepad 48.9_{\pm0.7} & \prepad 11.3_{\pm0.6} \\
CMF-pairwise-star & \prepad 16.7_{\pm0.1} & \prepad 37.5_{\pm1.0} & \prepad 48.3_{\pm0.7} & \prepad 11.3_{\pm0.6} \\
CMF-triplet & \prepad 17.8_{\pm1.2} & \prepad 37.7_{\pm1.4} & \prepad 47.9_{\pm1.2} & \prepad 12.0_{\pm1.0} \\
CMF-pairwise-star+ & \prepad 17.9_{\pm1.2} & \prepad 38.4_{\pm0.5} & \prepad 48.6_{\pm0.9} & \prepad 11.3_{\pm0.6} \\
\hline \hline
\end{tabular}
\end{minipage}
\begin{minipage}{0.4\linewidth}
\centering 
\footnotesize
% \setlength{\tabcolsep}{4pt}
\begin{tabular}{l | c c c c c } 
\hline \hline
\multicolumn{1}{c}{} & 
\multicolumn{5}{c}{Text $\implies$ Video}  \\
Method & R$@$1 & R$@$5 & R$@$10 & MdR & MnR  \\ 
\hline 
CCA  & - & - & - & - & 245.3  \\
JMDV  & - & - & - & - & 236.3  \\
VSE  & 12.3 & 30.1 & 42.3 & 14 & 57.7 \\
VSE++ & 15.4 & 39.6 & 53.0 & 9 & 43.8 \\
Multi. Cues  & 20.3 & 47.8 & 61.1 & $\textbf{6}$ & 28.3 \\

CE  & $\prepad 20.9_{\pm0.5}$ & $\prepad 49.7_{\pm0.4}$ & $\prepad 63.8_{\pm0.4}$ & $\prepad 5.7_{\pm0.6}$ & $\prepad 19.5_{\pm0.9}$ \\
\hline
% (No CMF) & \prepad 15.0_{\pm0.2} & \prepad 32.5_{\pm0.6} & \prepad 42.9_{\pm0.8} & \prepad 69.2_{\pm0.8} & \prepad 16.0_{\pm0.0} & \prepad 56.8_{\pm0.2} & \prepad 19.0_{\pm0.2} & \prepad 39.5_{\pm0.9} & \prepad 50.0_{\pm1.1} & \prepad 77.9_{\pm0.8} & \prepad 10.7_{\pm0.6} & \prepad 35.1_{\pm0.7}

% CE & $\prepad 00.0_{\pm0.0}$ & $\prepad 00.0_{\pm0.0}$ & $\prepad 00.0_{\pm0.0}$ & $\prepad 0.0_{\pm0.0}$ & $\prepad 00.0_{\pm0.0}$ \\
% AHOI  & $\prepad 00.0_{\pm0.0}$ & $\prepad 00.0_{\pm0.0}$ & $\prepad 00.0_{\pm0.0}$ & $\prepad 0.0_{\pm0.0}$ & $\prepad 00.0_{\pm0.0}$ \\
\hline
CMF & (still cooking) & & & \\
\hline \hline
\end{tabular}
\end{minipage}
\vspace{0.2cm}
\caption{\color{red}Text-to-Video retrieval results on the LSMDC dataset (left) and the MSVD dataset (right). $\dagger,\ddagger$ denote the winners of the 2016 and 2017 LSMDC challenges, respectively. For reference: MoEE and CCA~\cite{miech2018learning}, JSFusion~\cite{yu2018joint}, JMDV, CCA (right)~\cite{xu2015jointly}, Yu et al.~\cite{yu2016video}, VSE \cite{kiros2014unifying} (\cite{mithun2018learning}), VSE++ \cite{faghri2017vse} (\cite{mithun2018learning}), Multi. Cues \cite{mithun2018learning}  \color{black}}
\label{table:LSMDC-MSVD} 
\end{table*}



\subsection{Action Recognition}

\paragraph{Multi-Moments-in-Time:} This dataset is a recently released very-large scale collection of short videos for action recognition.  The Multi-Moments-in-Time dataset contains 313 classes, 1025862 training videos and 10000 validation videos \cite{monfortmoments}.

\subsection{Ablation Study: Module Design}

In this section, we assess the contribution of the proposed HOI module.  To do so, we conduct further experiments on the MSR-VTT dataset as well as the Mult-Moments-in-Time dataset.
\begin{table*}[h]
% \captionsetup{font=footnotesize}
\centering 
\footnotesize 
\setlength{\tabcolsep}{5pt}
\begin{tabular}{l | c c c c c | c c c c c} 
\hline \hline 
\multicolumn{1}{c}{} & 
\multicolumn{5}{c}{Text $\implies$ Video} & \multicolumn{5}{c}{Video $\implies$ Text} \\
Method & R$@$1 & R$@$5 & R$@$50 & MdR & MnR & R$@$1 & R$@$5 & R$@$50 & MdR & MnR \\ 

\hline
CMF-pairwise & $\prepad 31.6_{\pm0.3}$ & $\prepad 67.2_{\pm0.2}$ & $\prepad 97.0_{\pm0.2}$ & $\prepad 3.0_{\pm0.0}$ & $\prepad 9.9_{\pm0.1}$ & $\prepad 32.0_{\pm0.4}$ & $\prepad 68.9_{\pm1.1}$ & $\prepad 97.1_{\pm0.3}$ & $\prepad 3.0_{\pm0.0}$ & $\prepad 9.5_{\pm0.4}$ \\
CMF-pairwise-star & $\prepad 31.7_{\pm0.5}$ & $\prepad 66.9_{\pm0.6}$ & $\prepad 96.7_{\pm0.3}$ & $\prepad 3.0_{\pm0.0}$ & $\prepad 10.0_{\pm0.3}$ & $\prepad 32.7_{\pm0.4}$ & $\prepad 69.2_{\pm0.8}$ & $\prepad 96.9_{\pm0.1}$ & $\prepad 3.0_{\pm0.0}$ & $\prepad 9.4_{\pm0.3}$ \\
CMF-pairwise-star-tensor & $\prepad 32.3_{\pm0.3}$ & $\prepad 67.6_{\pm0.3}$ & $\prepad 97.2_{\pm0.1}$ & $\prepad 3.0_{\pm0.0}$ & $\prepad 9.50_{\pm0.2}$ & $\prepad 33.5_{\pm0.7}$ & $\prepad 69.2_{\pm1.2}$ & $\prepad 97.4_{\pm0.1}$ & $\prepad 3.0_{\pm0.0}$ & $\prepad 8.9_{\pm0.6}$ \\
CMF-triplet & $\prepad 30.5_{\pm0.2}$ & $\prepad 66.5_{\pm0.7}$ & $\prepad 96.8_{\pm0.1}$ & $\prepad 3.0_{\pm0.0}$ & $\prepad 9.8_{\pm0.2}$ & $\prepad 32.0_{\pm1.4}$ & $\prepad 67.5_{\pm0.8}$ & $\prepad 96.8_{\pm0.4}$ & $\prepad 3.0_{\pm0.0}$ & $\prepad 9.4_{\pm0.6}$ \\
% CMF-pairwise-star-random & $\prepad 27.4_{\pm0.7}$ & $\prepad 62.6_{\pm1.4}$ & $\prepad 96.1_{\pm0.1}$ & $\prepad 3.0_{\pm0.0}$ & $\prepad 11.3_{\pm0.3}$ & $\prepad 29.6_{\pm0.6}$ & $\prepad 65.5_{\pm1.0}$ & $\prepad 96.8_{\pm0.1}$ & $\prepad 3.0_{\pm0.0}$ & $\prepad 10.2_{\pm0.2}$ \\
CMF-pairwise-star-specific & $\prepad 31.1_{\pm0.7}$ & $\prepad 65.9_{\pm1.4}$ & $\prepad 96.8_{\pm0.1}$ & $\prepad 3.0_{\pm0.0}$ & $\prepad 10.1_{\pm0.3}$ & $\prepad 31.9_{\pm0.6}$ & $\prepad 67.6_{\pm1.0}$ & $\prepad 96.9_{\pm0.1}$ & $\prepad 3.0_{\pm0.0}$ & $\prepad 9.5_{\pm0.2}$ \\

\hline \hline
\end{tabular}
% \vspace{0.3cm}
\caption{\color{red}Comparison of paragraph-video retrieval methods trained with video-level information on the ActivityNet-captions dataset (val1 test-split). For reference: LSTM-YT~\cite{venugopalan2015sequence} (\cite{zhang2018cross}), NOCTXT~\cite{venugopalan2014translating} (\cite{zhang2018cross}), HSE(4SEGS)~\cite{zhang2018cross}, DENSE~\cite{krishna2017dense}, FSE~\cite{zhang2018cross} \color{black}}
\label{table:activity-net} 
\end{table*}
\subsection{Ablation Study: Temporal Aggregation}

\subsection{Computation Burden Analysis}